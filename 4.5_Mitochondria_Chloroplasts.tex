\newpage

\StudyTitle{Mitochondria and Chloroplasts (Chapter 4.5)}

Section 4.5 focuses on the organelles responsible for cellular energy transformations: \textbf{mitochondria} and \textbf{chloroplasts}. Emphasis is placed on how their structure reflects their function and how multiple lines of evidence support their evolutionary origin through endosymbiosis.

\begin{multicols}{2}

\section*{Mitochondria: Cellular Powerhouses}

\begin{center}
\begin{minipage}{0.85\linewidth}
  \centering
  \includegraphics[width=\linewidth]{images/mitochondria.jpg}
\end{minipage}
\end{center}

  
Mitochondria are the primary sites of ATP production in eukaryotic cells. Most cellular ATP is generated through metabolic reactions that occur within this organelle.

Key structural features include:
\begin{itemize}
  \item A \textbf{double membrane}
  \item A smooth outer membrane
  \item A highly folded inner membrane forming \textbf{cristae}
  \item An internal fluid-filled space called the \textbf{matrix}
  \item A compartment between membranes called the \textbf{intermembrane space}
\end{itemize}

The extensive folding of the inner membrane increases surface area, allowing more proteins involved in energy metabolism to be embedded in the membrane.

\section*{Mitochondrial Function}

Proteins located on the inner membrane and within the matrix carry out the reactions that lead to ATP synthesis. These reactions involve the controlled transfer of energy through metabolic pathways rather than direct combustion of fuels.

Because ATP is required for nearly all cellular processes, mitochondria are essential for the survival of most eukaryotic cells.

Mitochondria are not synthesized de novo. Instead, they divide independently within the cell, and their numbers increase prior to cell division. Most proteins required for mitochondrial structure and division are encoded by genes in the nucleus.

\section*{Chloroplasts: Energy Conversion via Light}
 

Chloroplasts are found in plants and some protists and are responsible for capturing light energy and converting it into chemical energy.

Structural features include:
\begin{itemize}
  \item A \textbf{double outer membrane}
  \item An internal system of flattened membranes called \textbf{thylakoids}
  \item Stacks of thylakoids known as \textbf{grana}
  \item An internal fluid-filled space called the \textbf{stroma}
\end{itemize}

Light-dependent reactions occur on the thylakoid membranes, while the synthesis of sugars occurs in the stroma.

\begin{center}
    

 \begin{minipage}{0.85\linewidth}
    \centering
    \includegraphics[width=\linewidth]{images/chloroplasts.jpg}
  \end{minipage}
  \end{center}
\section*{Functional Parallels Between Organelles}

Despite their different roles, mitochondria and chloroplasts share several important characteristics:
\begin{itemize}
  \item Both generate ATP
  \item Both contain their own DNA
  \item Both divide independently within the cell
\end{itemize}

These shared features suggest a common evolutionary origin.

\section*{Endosymbiotic Theory}

The endosymbiotic theory proposes that mitochondria and chloroplasts originated as free-living prokaryotic cells that were engulfed by an ancestral host cell.

Evidence supporting this theory includes:
\begin{itemize}
  \item The presence of double membranes
  \item Circular DNA similar to bacterial chromosomes
  \item Independent replication within the cell
\end{itemize}

Over time, these engulfed cells formed stable, mutually beneficial relationships with their hosts and became permanent organelles. Without these organelles, the energetic demands of complex life would not be sustainable.

\end{multicols}


 





