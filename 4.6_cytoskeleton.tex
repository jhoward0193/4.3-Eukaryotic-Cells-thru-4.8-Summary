\newpage

\StudyTitle{The Cytoskeleton (Chapter 4.6)}

Section 4.6 introduces the \textbf{cytoskeleton}, a dynamic network of protein fibers that provides structural support, organizes the cell interior, and enables movement of both cells and intracellular components. Emphasis is placed on the cytoskeleton as an active, adaptable system rather than static scaffolding.

\begin{multicols}{2}

\begin{center}
\begin{minipage}{0.85\linewidth}
  \centering
  \includegraphics[width=\linewidth]{images/cytoskeleton.png}
\end{minipage}
\end{center}

\section*{Overview of the Cytoskeleton}

The cytoskeleton is composed of crisscrossing protein fibers that:
\begin{itemize}
  \item Maintain cell shape
  \item Anchor and organize organelles
  \item Enable intracellular transport
  \item Drive cell movement and division
\end{itemize}

Eukaryotic cells possess three major types of cytoskeletal filaments, each with distinct structures and functions.

\section*{Microfilaments (Actin Filaments)}

Microfilaments are long, thin fibers composed of the protein \textbf{actin}. They are especially important in cell shape and movement.

Functions include:
\begin{itemize}
  \item Cell crawling and migration
  \item Formation of the cell cortex beneath the plasma membrane
  \item Muscle contraction when interacting with myosin
\end{itemize}

During cell crawling, actin polymerization pushes the plasma membrane forward, while myosin contracts actin filaments to pull the cell body toward the leading edge. This same actin–myosin interaction is later adapted in muscle cells to produce contraction.

\section*{Microtubules}

Microtubules are hollow cylinders composed of tubulin proteins. They are the largest and most rigid cytoskeletal elements and serve as structural tracks within the cell.

Key roles include:
\begin{itemize}
  \item Transport of vesicles and organelles
  \item Formation of the mitotic spindle during cell division
  \item Structural support for cilia and flagella
\end{itemize}

Microtubules originate from organizing centers such as the centrosome and extend outward through the cytoplasm.

\section*{Intermediate Filaments}

Intermediate filaments are composed of various fibrous proteins, including keratin and vimentin. Their primary role is mechanical strength rather than movement.

Functions include:
\begin{itemize}
  \item Resistance to mechanical stress
  \item Maintenance of cell integrity
  \item Stabilization of cell shape over time
\end{itemize}

Unlike microfilaments and microtubules, intermediate filaments are relatively stable and less dynamic.

\section*{Molecular Motors and Intracellular Transport}

The cytoskeleton functions as a system of tracks for motor proteins that move materials within the cell.

Important motor proteins include:
\begin{itemize}
  \item \textbf{Kinesin}, which typically transports cargo along microtubules toward the cell periphery
  \item \textbf{Dynein}, which generally moves cargo toward the cell center
  \item \textbf{Myosin}, which moves along actin filaments
\end{itemize}

These motor proteins use energy from ATP hydrolysis to transport vesicles, organelles, and other cargo, functioning analogously to trains moving along rails.

\section*{Centrosomes and Cell Division}

Centrosomes act as microtubule-organizing centers in animal cells and typically contain a pair of centrioles. They play a critical role during cell division by helping assemble the mitotic spindle that separates chromosomes into daughter cells.

\section*{Functional Significance of the Cytoskeleton}

The cytoskeleton allows cells to:
\begin{itemize}
  \item Change shape in response to signals
  \item Move toward or away from stimuli
  \item Organize internal components efficiently
  \item Divide accurately during reproduction
\end{itemize}

Because these functions are essential for life, defects in cytoskeletal components often lead to serious cellular and organismal dysfunction.

\end{multicols}
