
\newpage

\StudyTitle{Extracellular Structures and Cell Movement (Chapter 4.7)}

Section 4.7 examines how cells move and interact with their surroundings. The lecture emphasized that cellular movement can be driven by internal cytoskeletal forces or by external structures, and that animal cells are embedded within an extracellular matrix that actively influences cell behavior.

\begin{multicols}{2}

\section*{Modes of Cellular Movement}

Cells move using different strategies depending on cell type and environment. Two major categories of movement were emphasized:
\begin{itemize}
  \item \textbf{Cell crawling}, driven by internal cytoskeletal proteins
  \item \textbf{Surface appendages}, such as cilia or flagella
\end{itemize}

These mechanisms rely on different molecular systems but share the common goal of repositioning cells or moving fluids.

\section*{Cell Crawling}

Cell crawling occurs when actin filaments polymerize beneath the plasma membrane, pushing it forward. At the same time, myosin motors pull the cell body in the same direction.

This process is especially important during:
\begin{itemize}
  \item Development
  \item Wound healing
  \item Immune cell movement
\end{itemize}

The actin–myosin interaction used in cell crawling is also adapted for muscle contraction, where highly ordered arrays of these proteins generate force at the tissue level.

\section*{Cilia and Flagella in Eukaryotic Cells}

Eukaryotic cilia and flagella are external structures used for movement or fluid transport. They share a characteristic internal arrangement of microtubules known as the \textbf{9 + 2 structure}.

\begin{center}
\begin{minipage}{0.85\linewidth}
  \centering
  \includegraphics[width=\linewidth]{images/9.2flag.jpg}
\end{minipage}
\end{center}

Key features include:
\begin{itemize}
  \item Bundles of microtubules surrounded by membrane
  \item Whip-like, undulating motion
  \item A basal body that anchors the structure to the cell
\end{itemize}

Cilia are typically short and numerous, while flagella are longer and fewer in number.


\section*{Prokaryotic Flagella}

Prokaryotic flagella differ fundamentally from eukaryotic flagella in both structure and energy source. Bacterial flagella rotate like propellers and are powered by proton gradients across the plasma membrane.

Archaeal motility structures, called \textit{archaella}, resemble bacterial flagella in appearance but are powered by ATP hydrolysis. These similarities reflect \textbf{convergent evolution} rather than shared ancestry.

\section*{Plant Cell Walls}

Plant cells possess rigid cell walls composed primarily of cellulose fibers. These walls:
\begin{itemize}
  \item Provide protection and mechanical support
  \item Maintain cell shape
  \item Resist osmotic pressure
\end{itemize}

Adjacent plant cells are held together by the middle lamella, contributing to tissue integrity.

\section*{Extracellular Matrix (ECM) in Animal Cells}

Animal cells do not have cell walls. Instead, they secrete a complex network of glycoproteins known as the \textbf{extracellular matrix} (ECM).

Major components include:
\begin{itemize}
  \item Collagen
  \item Elastin
  \item Fibronectin
  \item Proteoglycans
\end{itemize}

The ECM provides structural support, strength, and elasticity to tissues.

\section*{Integrins and Cell–ECM Interactions}

Integrins are transmembrane proteins that link the ECM to the cytoskeleton. Through these connections, integrins allow the ECM to influence cell behavior.

Integrin signaling can:
\begin{itemize}
  \item Alter gene expression
  \item Influence cell migration
  \item Coordinate mechanical and chemical signals
\end{itemize}

In this way, the ECM is not merely structural but plays an active role in regulating cellular function.

\section*{Functional Significance}

Extracellular structures and movement systems allow cells to:
\begin{itemize}
  \item Respond dynamically to their environment
  \item Coordinate behavior within tissues
  \item Maintain tissue structure under mechanical stress
\end{itemize}

These systems are essential for multicellular organization and function.



\end{multicols}