
\newpage
\StudyTitle{The Endomembrane System (Chapter 4.4)}

Section 4.4 introduces the \textbf{endomembrane system}, a network of membranes that divides the cytoplasm of eukaryotic cells into functional compartments. The central idea emphasized in lecture is that compartmentalization allows the cell to carry out complex chemical processes efficiently and in a regulated manner.

\begin{multicols}{2}

\section*{Overview of the Endomembrane System}

The endomembrane system includes membranes that are either physically connected or exchange materials through vesicles. Together, these structures coordinate the synthesis, modification, transport, and degradation of macromolecules.

Major components include:
\begin{itemize}
  \item Nuclear envelope
  \item Endoplasmic reticulum (rough and smooth)
  \item Golgi apparatus
  \item Lysosomes
  \item Transport vesicles
  \item Vacuoles (in plants and some protists)
\end{itemize}

\section*{Rough Endoplasmic Reticulum (RER)}

The rough ER is distinguished by ribosomes attached to its cytoplasmic surface. These ribosomes synthesize:
\begin{itemize}
  \item Proteins destined for secretion
  \item Proteins inserted into membranes
  \item Proteins targeted to specific organelles
\end{itemize}

As proteins are synthesized, they enter the lumen of the RER, where they may begin folding and undergo early chemical modification.

\section*{Smooth Endoplasmic Reticulum (SER)}

The smooth endoplasmic reticulum lacks bound ribosomes and contains embedded enzymes that support several key cellular functions:
\begin{itemize}
  \item Synthesis of lipids, phospholipids, and steroid hormones
  \item Carbohydrate metabolism
  \item Detoxification of drugs and poisons, especially in liver cells
  \item Storage and regulated release of calcium ions (Ca$^{2+}$)
\end{itemize}

In muscle cells, specialized smooth ER (the sarcoplasmic reticulum) plays a critical role in contraction by releasing and reabsorbing Ca$^{2+}$ ions, illustrating how the same organelle can be functionally specialized in different cell types.


\section*{Golgi Apparatus}

The Golgi apparatus functions as the cell’s central processing, sorting, and distribution center for proteins and lipids received from the endoplasmic reticulum.

Its roles include:
\begin{itemize}
  \item Chemical modification of proteins
  \item Sorting macromolecules by their cellular destination
  \item Packaging materials into transport vesicles
  \item Contributing to the synthesis and renewal of cellular membranes
\end{itemize}

The Golgi apparatus has a distinct polarity: the \textit{cis} face receives vesicles from the ER, while the \textit{trans} face directs vesicles to specific destinations, such as the plasma membrane, lysosomes, or secretion outside the cell.

\section*{Lysosomes}

Lysosomes are membrane-bound organelles that originate from the Golgi apparatus and contain digestive enzymes.

These enzymes:
\begin{itemize}
  \item Break down macromolecules such as proteins, nucleic acids, lipids, and carbohydrates
  \item Recycle components of old or damaged organelles
  \item Function optimally at an acidic pH
\end{itemize}

By isolating these enzymes within membranes, the cell prevents accidental damage to other cellular structures. Uncontrolled release of lysosomal enzymes could damage or destroy the cell.

\section*{Vacuoles}

Vacuoles are most prominent in plant cells, where a large central vacuole may occupy most of the cell’s volume.

Functions include:
\begin{itemize}
  \item Storage of water, ions, and metabolites
  \item Regulation of internal pressure (turgor)
  \item Contribution to cell growth
\end{itemize}

The vacuolar membrane, called the \textbf{tonoplast}, contains channels that regulate the movement of water and solutes, helping maintain cellular tonicity and osmotic balance. Some fungi and protists also use vacuoles for water balance.

\section*{Lipid Droplets and Microbodies}

Lipid droplets store neutral lipids and consist of lipid cores surrounded by a phospholipid monolayer. These structures provide an efficient way to store energy-rich molecules.

Microbodies are a diverse group of organelles that include \textbf{peroxisomes}. Peroxisomes contain enzymes that:
\begin{itemize}
  \item Oxidize fatty acids
  \item Participate in detoxification reactions
  \item Utilize peroxide in metabolic reactions
\end{itemize}

Peroxisomes contain the enzyme catalase, which decomposes hydrogen peroxide according to the reaction:
\[
2 H_2O_2 \rightarrow 2 H_2O + O_2
\]




\end{multicols}
