\section*{Key Takeaways from Section 4.3}

\begin{itemize}
  \item Eukaryotic cells are defined by membrane-bound compartments
  \item The nucleus isolates and regulates genetic information
  \item Ribosomes are universal and essential
  \item Compartmentalization increases efficiency and regulatory control
  \item The cytoskeleton enables structure, transport, and movement
\end{itemize}

\section*{Key Takeaways from Section 4.4}

\begin{itemize}
  \item The endomembrane system divides the cell into functional compartments
  \item RER and SER serve distinct but complementary roles
  \item The Golgi apparatus sorts and packages macromolecules
  \item Lysosomes enable safe intracellular digestion
  \item Vacuoles, lipid droplets, and peroxisomes support storage and metabolism
\end{itemize}

\section*{Key Takeaways from Section 4.5}

\begin{itemize}
  \item Mitochondria generate most cellular ATP
  \item Chloroplasts convert light energy into chemical energy
  \item Structure and function are tightly linked in both organelles
  \item Both organelles arose via endosymbiosis
  \item Energy-processing organelles underpin eukaryotic complexity
\end{itemize}

\section*{Key Takeaways from Section 4.6}

\begin{itemize}
  \item The cytoskeleton is a dynamic, functional network
  \item Actin filaments drive cell crawling and muscle contraction
  \item Microtubules serve as structural tracks and division machinery
  \item Intermediate filaments provide mechanical strength
  \item Motor proteins convert chemical energy into movement
\end{itemize}

\section*{Key Takeaways from Section 4.7}

\begin{itemize}
  \item Cell movement can be driven internally or by external appendages
  \item Actin and myosin power cell crawling and muscle contraction
  \item Eukaryotic cilia and flagella share a 9 + 2 structure
  \item Prokaryotic motility structures evolved independently
  \item The ECM actively influences cell behavior through integrins
\end{itemize}

\section*{Key Takeaways from Section 4.8}

\begin{itemize}
  \item Cell identity is determined by surface proteins and glycolipids
  \item MHC proteins enable self–nonself recognition
  \item Cell junctions hold tissues together and coordinate function
  \item Tight junctions form selective barriers
  \item Communicating junctions allow direct cell-to-cell signaling
\end{itemize}